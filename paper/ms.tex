\documentclass[modern]{aastex62}

\usepackage{graphicx}
\usepackage{xcolor}
\usepackage{xspace}
\usepackage[sort&compress]{natbib}
\usepackage[hang,flushmargin]{footmisc}


% style tweaks
\newcommand{\acronym}[1]{{\small{#1}}}
\newcommand{\project}[1]{\textsl{#1}}
\newcommand{\code}[1]{{\texttt{#1}}}
\newcommand{\todo}[1]{\textcolor{red}{#1}}

% the following is stolen from Adrian Price-Whelan (github.com/adrn/latex-init):
\usepackage{hyperref}
\definecolor{niceblue}{rgb}{0.0, 0.4, 0.65}
\definecolor{linkcolor}{rgb}{0.02,0.35,0.55}
\definecolor{citecolor}{rgb}{0.4,0.4,0.4}
\hypersetup{colorlinks=true,linkcolor=linkcolor,citecolor=citecolor,
            filecolor=linkcolor,urlcolor=linkcolor}
\hypersetup{pageanchor=false}

% astronomy
\newcommand{\teff}{\ensuremath{T_{\rm eff}}}
\newcommand{\logg}{\ensuremath{\log g}}
\newcommand{\feh}{\ensuremath{\mathrm{[Fe/H]}}}
\newcommand{\vt}{\ensuremath{v_t}}
\newcommand{\mh}{\ensuremath{\mathrm{[M/H]}}}
\newcommand{\xh}{\ensuremath{\mathrm{[X/H]}}}
\newcommand{\I}{\textsc{I}}
\newcommand{\II}{\textsc{II}}
\newcommand{\vsini}{\ensuremath{v \sin{i}}}
\newcommand{\gcm}{\ensuremath{\mathrm{g}~\mathrm{cm}^{-3}}}
\newcommand{\kms}{\ensuremath{\mathrm{km}~\mathrm{s}^{-1}}}
\newcommand{\masyr}{\ensuremath{\mathrm{mas}~\mathrm{yr}^{-1}}}
\newcommand{\msun}{\ensuremath{\mathrm{M}_\odot}}
\newcommand{\ang}{\text{\normalfont\AA}}


\newcommand{\TF}{\code{TensorFlow}\xspace}
\newcommand{\python}{\code{python}\xspace}
\newcommand{\HARPS}{\project{\acronym{HARPS}}\xspace}
\newcommand{\HIRES}{\project{\acronym{HIRES}}\xspace}
\newcommand{\RV}{\acronym{RV}\xspace}
\newcommand{\RVs}{\acronym{RV}s\xspace}


% stolen from Ben Pope:
\newcommand{\kepler}{\emph{Kepler}\xspace}
\newcommand{\hipparcos}{\emph{Hipparcos}\xspace}
\newcommand{\gaia}{\emph{Gaia}\xspace}
\newcommand{\ktwo}{\emph{K2}\xspace}
\newcommand{\TESS}{\emph{\acronym{TESS}}\xspace}

% misc shortcuts
\newcommand{\flatiron}{Flatiron Institute, Simons Foundation, 162 Fifth Ave, New York, NY 10010, USA}
\newcommand{\chicago}{Department of Astronomy and Astrophysics, University of
Chicago, 5640 S. Ellis Ave, Chicago, IL 60637, USA}
\newcommand{\USP}{Universidade de Sao Paulo}
\newcommand{\MIT}{MIT}

\newcommand{\hoststar}{\acronym{HIP}\ 96160\xspace}

\shorttitle{Working Title}
\shortauthors{Bedell et al.}

\setlength{\parindent}{1.4em} % trust in Hogg
\begin{document}\sloppy\sloppypar\raggedbottom\frenchspacing % trust in Hogg

\graphicspath{ {figures/} }
\DeclareGraphicsExtensions{.pdf,.eps,.png}

\title{A warm sub-Neptune transiting a Solar twin in TESS}

\author[0000-0001-9907-7742]{Megan Bedell}
\affiliation{\flatiron}

\author{Dan Foreman-Mackey}
\affiliation{\flatiron}

\author{Chelsea Huang}
\affiliation{\MIT}

\author{Jennifer Burt}
\affiliation{\MIT}

\author{Jhon Yana Galarza}
\affiliation{\USP}

\author{Jorge Melendez}
\affiliation{\USP}

\author{Lorenzo Spina}
\affiliation{Monash}

%\author{Jacob L. Bean}
%\affiliation{\chicago}


\author{\todo{other STPS folks}}



\correspondingauthor{Megan Bedell}
\email{mbedell@flatironinstitute.org}


\begin{abstract}\noindent
% Precise characterization of exoplanet host stars is critical to our understanding of their properties. 
% However, stellar spectroscopic characterization can be plagued with 
% context
% aims
% methods
% results
\end{abstract}

\keywords{
}

\section{Introduction}
\label{s:intro}

%The Transiting Exoplanet Survey Satellite (\TESS) has detected over one thousand candidate exoplanets to date. (blah blah)

%\hoststar has been studied extensively through a dedicated \RV planet search and spectroscopic abundance survey targeting solar twin stars. ...

\section(Data)
\label{s:data}

\subsection{Photometry}

(about \TESS photometry)

\subsection{Spectra}

(about \HARPS \RVs)

(about co-added spectrum - SNR etc)

\section{Analysis \& Results}
\label{s:analysis}

\subsection{Stellar Characterization}
\label{s:analysis:star}


(cite spectroscopic properties + abundances from previous work, plus Gaia DR2)
(isochrones fit - we use this mass \& radius going forward)

(attempts at seismology \& rotation period)

\subsection{Photometric Analysis}
\label{s:analysis:photometry}


\subsection{\RV Analysis}
\label{s:analysis:rvs}


\subsection{Joint Analysis}
\label{s:analysis:joint}


\section{Discussion}
\label{s:discussion}


(Low probability of background contaminants - discuss Gaia, spectra)

(reliability of the RV fit - potential for more planets? activity?)

(M-R diagram)

(prospects for further compositional analysis)

(prospects for atmospheric characterization)

\section{Conclusion}
\label{s:conclusion}


\acknowledgements
We thank ...

\software{
    \code{Astropy} \citep{astropy},
    \code{exoplanet} \citep{exoplanet},
    \code{IPython} \citep{ipython},
    \code{matplotlib} \citep{matplotlib},
    \code{numpy} \citep{numpy},
    \code{scipy} (\url{https://www.scipy.org/}),
}

\facility{TESS, HARPS}

\bibliographystyle{aasjournal}
\bibliography{ms}

\end{document}